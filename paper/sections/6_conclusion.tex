%%%%%%%%%%%%%%%%%%%%%%%%%%%%%%%%%%%%%%%%%%%%%%%%%%%%%%%%%%%%%%
% Conclusion
%%%%%%%%%%%%%%%%%%%%%%%%%%%%%%%%%%%%%%%%%%%%%%%%%%%%%%%%%%%%%%
\documentclass[../paper.tex]{subfiles}
\begin{document}
    In this work, we have presented \texttt{GraphLab.jl}, a comprehensive and extensible framework for graph partitioning,
    designed to support both research and education in graph theory and partitioning problems.
    By integrating foundational partitioning techniques --- such as coordinate, inertial, and spectral
    bisection, as well as random spheres, space-filling curves, and nested
    dissection --- together utilities for visualization, benchmarking, and quality assessment, the framework provides
    a structured and interactive environment for analyzing and comparing graph partitioning strategies.
    
    A key strength of \texttt{GraphLab.jl} lies in its modular design and extensibility. Users can experiment
    with a diverse range of partitioning algorithms, leveraging both native implementations and external software, all within a unified and reproducible setting.
    % The possibility to integrate with \texttt{Pluto.jl} or other notebook interfaces further enhances interactivity, making the framework especially well suited for teaching and exploratory research.
    
    % Ongoing developments aim to broaden the framework's capabilities.
    % Planned features include additional evaluation metrics, such as modularity, to provide more comprehensive assessments of partition quality.
    % Support for further partitioning techniques, particularly label propagation-based methods, is also under active development, alongside performance optimizations to better support large-scale graphs.
    Ongoing developments aim to broaden the framework's capabilities, with planned extensions to partitioning techniques, evaluation metrics, and overall performance.
\end{document}