%%%%%%%%%%%%%%%%%%%%%%%%%%%%%%%%%%%%%%%%%%%%%%%%%%%%%%%%%%%%%%
% Conclusion
%%%%%%%%%%%%%%%%%%%%%%%%%%%%%%%%%%%%%%%%%%%%%%%%%%%%%%%%%%%%%%
% \documentclass[../paper.tex]{subfiles}
% \begin{document}
    In this work, we have presented \texttt{GraphLab.jl}, a
    framework for graph partitioning
    designed to support research and education in graph theory
    and partitioning problems.
    It incorporates fundamental partitioning methods, such as
    coordinate, inertial, and spectral bisection, as well as
    random spheres, space-filling curves, and nested
    dissection.  It additionally offers utilities for
    visualization, benchmarking, and partition quality
    assessment, in an effort to provide a unified environment for
    analyzing and comparing graph partitioning algorithms.
    Ongoing developments aim to broaden the framework's
    capabilities with implementations of additional
    partitioning techniques, of multilevel
    methods~\cite{10938528, DBLP:conf/wea/SandersS13}, and by
    enabling parallel execution~\cite{7859409}, particularly in
    the recursive implementation of geometric-based algorithms.
% \end{document}

    % A key strength of \texttt{GraphLab.jl} lies in its modular design and extensibility.
    % Users can experiment
    % with a diverse range of partitioning algorithms, leveraging both native implementations and external software.
    % The possibility to integrate with \texttt{Pluto.jl} or other notebook interfaces further enhances interactivity, making the framework especially well suited for teaching and exploratory research.
    
    % Ongoing developments aim to broaden the framework's capabilities.
    % Planned features include additional evaluation metrics, such as modularity, to provide more comprehensive assessments of partition quality.
    % Support for further partitioning techniques, particularly label propagation-based methods, is also under active development, alongside performance optimizations to better support large-scale graphs.