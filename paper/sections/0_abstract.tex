%%%%%%%%%%%%%%%%%%%%%%%%%%%%%%%%%%%%%%%%%%%%%%%%%%%%%%%%%%%%%%
% Abstract
%%%%%%%%%%%%%%%%%%%%%%%%%%%%%%%%%%%%%%%%%%%%%%%%%%%%%%%%%%%%%%
    We design and implement \texttt{GraphLab.jl}, a \texttt{Julia} package facilitating the study, experimentation, and research of
    graph partitioning. \texttt{GraphLab.jl} explores the principles and trade-offs of partitioning
    algorithms. It offers a set
    of methods, including coordinate, inertial, and spectral
    bisection, random spheres, space-filling curves, and nested
    dissection, with support for recursive partitioning.
    The
    package includes routines for generating adjacency matrices, 
    computing partition quality metrics, benchmarking problems,
    and visualizing partitioned graphs. 
    % Our work provides a \texttt{Julia}-based framework to learners and researchers engaging in
    % graph theory and related partitioning problems. 
    % \textcolor{red}{AE NOTE :  What do you mean "growing set" ? How is it growing? we are just introducing it. I would remove this.}

    % \textcolor{red}{AE NOTE: "routines for benchmarking problems" or routines for "benchmark problems"?}
    % \texttt{GraphLab.jl}
    % enables integration with external graph partitioning software, thus allowing users to compare additional methods and results in a unified environment. 
    % \textcolor{red}{AE NOTE: You are starting the sentence two times with "GraphLab.jl". }
    % \textcolor{red}{AE NOTE:  Don't start with "last" if you have not stated "first" or "second"  above, maybe you mean "finally"?  This sentence is somewhat redundant and overlaps with the first sentence of the abstract, which already states that it "facilitates the study, experimentation, and research of graph partitioning". }



% Old abstract

    % \texttt{GraphLab.jl} is a Julia package designed to facilitate the study, experimentation, and research of graph partitioning. \texttt{GraphLab.jl} provides a framework for exploring the principles and trade-offs of partitioning algorithms through hands-on tools. It implements a growing set of methods—including coordinate, inertial, and spectral bisection, random spheres, space-filling curves, and nested dissection—with support for recursive partitioning. The package includes routines for generating adjacency matrices, computing partition quality metrics, benchmarking problems, and visualizing partitioned graphs. \texttt{GraphLab.jl} enables integration with external graph partitioning software, thus allowing users to compare additional methods and results in a unified environment. This work also aims to introduce Julia's capabilities to learners and researchers engaging in graph theory and related partitioning problems.



% Vey Old abstract

    % Partitioning computational problems into smaller problems that can be
    % efficiently solved in parallel is fundamental in high-performance computing
    % (HPC). Effective partitioning requires balancing subproblem sizes to
    % ensure load distribution while minimizing inter-problem communication.
    % We develop a comprehensive framework that streamlines the study and exploration of graph partitioning. It provides implementations of fundamental partitioning algorithms, visualization tools, adjacency matrix construction utilities, partition quality metrics, and benchmarking capabilities.
    % Additionally, the framework integrates with popular graph partitioning
    % software, facilitating systematic investigation, experimentation and visualization in graph partitioning.